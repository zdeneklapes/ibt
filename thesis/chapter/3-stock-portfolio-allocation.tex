\documentclass[../xlapes02]{subfiles}
\begin{document}
    \chapter{Stock Portfolio Allocation}\label{sec:stock-portfolio-allocation}
    The goal of this chapter is to describe the stock portfolio allocation problem in \cref{sec:portfolio-management-task}, and the environment, defined in \cref{sec:environment} that we have created to solve it. We will also describe the datasets that we have used to train and test our agent, in \cref{sec:data-engineering}.


    \section{Portfolio Management Task}\label{sec:portfolio-management-task}
    Let's consider a portfolio consisting of $N$ risky assets traded over $T$ time slots. The objective of portfolio management is to maximize profits while minimizing risks. We denote the closing prices of all assets at time slot $t$ as $\mathbf{p}(t) \in \mathbb{R}^N$. The price relative vector $\mathbf{y}(t) \in \mathbb{R}^N$ is defined as the element-wise division of $\mathbf{p}(t)$ by $\mathbf{p}(t-1)$, representing the relative change in prices from one-time slot to the next~\cite{finrl-portfolio-allocation-2020}:

    \begin{equation}
        \mathbf{y}(t)=\left[\frac{\mathbf{p_1}(t)}{\mathbf{p_1}(t-1)}, \frac{\mathbf{p_2}(t)}{\mathbf{p_2}(t-1)}, \cdots, \frac{\mathbf{p_N}(t)}{\mathbf{p_N}(t-1)}, 1\right]\text{ , for }t=1,\ldots,T
    \end{equation}

    where $\mathbf{p}(0) \in \mathbb{R}^N$ is the vector of opening prices at $t = 1$. Let $\mathbf{w}(t) \in \mathbb{R}^N$ denote the portfolio weights for each of $N$ risky assets, which are updated at the beginning of time slot $t$. Let $v(t) \in \mathbb{R}$ denote the portfolio value at the beginning of time slot $t+1$, ignoring transaction costs. The relative portfolio value is defined as the ratio between the portfolio value at the end of time slot $t$ and that at the beginning of time slot $t$~\cite{finrl-portfolio-allocation-2020}:

    \begin{equation}
        v(t)=v(t-1)(\mathbf{w}(t)^\top\mathbf{y}(t))\text{ , for }t=1,\ldots,T
    \end{equation}

    where $v(0)$ is the initial capital, $v(t-1)$ is the portfolio value at the end of time slot $t-1$ according to the previous realocation (using $v(t-2)$ and weights $w(t-1)$), and $\mathbf{w}(t)^\top\mathbf{y}(t)$ is the portfolio change according to the new weights. The rate of portfolio return is defined as the ratio between the portfolio value at the end of time slot $t-1$ and that at the end of time slot $t$~\cite{finrl-portfolio-allocation-2020}:
    \begin{equation}
        \rho(t)=\frac{v(t)}{v(t-1)}-1=\mathbf{w}(t)^\top\mathbf{y}(t)-1\text{ , for }t=1,\ldots,T
    \end{equation}

    The risk of a portfolio is defined as the variance of the rate of portfolio return $\varrho(t)$:
    \begin{equation}
        \text{Risk}(t)=\text{Var}(\rho(t))=\text{Var}(\mathbf{w}(t)^\top \mathbf{y}(t) - 1)
    \end{equation}

    If there is no transaction cost, the final portfolio value is:
    \begin{equation}
        v(t) = v(0) \prod_{t=2}^T\mathbf{w}(t)^\top \mathbf{y}(t)
    \end{equation}

    The portfolio management task~\cite{enwiki:1043516653} involves finding an optimal portfolio weight vector $\mathbf{w}^*(t) \in \mathbb{R}^N$ at time slot $t$ such that $\mathbf{w}^*(t)$ is the argmax of the objective function:
    \begin{equation}
        w^*(t)=\argmax_{w(t)}\mathbf{w}^\top(t)\mathbf{y}(t)-\lambda\mathbf{w}^\top(t)\text{Cov}(\mathbf{y}(t))\mathbf{w}(t),
    \end{equation}
    where $\lambda > 0$ is the risk aversion parameter, $\mathbf{y}(t) \in \mathbb{R}^N$ is the estimated price relative vector at time slot $t$ based on a regression model using predictive financial features and the Capital Asset Pricing Model (CAPM)~\cite{fama-2004}, and $\text{Cov}(\mathbf{y}(t))$ is the estimated sample covariance matrix at time slot $t$ using historical data. The objective is to maximize the portfolio's expected return while considering the risk (variance) of the portfolio, with the constraint that the portfolio weights sum up to 1 and are bounded between 0 and 1 for each asset~\cite{finrl-portfolio-allocation-2020}:
    \begin{equation}
        \sum_{i=1}^{N}w_i(t)=1,\quad w_i(t)\in[0,1],\quad t=1,\ldots,T
    \end{equation}


    \section{Environment}\label{sec:environment}
    In this section, we describe our approach to environment modeling by first defining the action space \cref{subsec:action-space}, state and observation space~\cref{subsec:state-space}, and reward function~\cref{subsec:reward-function}. The theory behind this was already described in~\cref{ch:reinforcement-learning}, so now we will focus on the practical implementation of the environment.

    \subsection{State and Observation Space}\label{subsec:state-space}
    In portfolio Management, \emph{State space} and \emph{Observation space} is the same. It is defined as the matrix $N\times F$, where $N$ is the number of risky assets (companies) and $F$ is the number of features. The features are the financial fundamental or technical indicators. The state space is the matrix of the elements of the risky assets at time $t$. The observation space is the matrix of the features of the risky assets at time $t+1$, more about Data Engineering in~\cref{sec:data-engineering}.

    \subsection{Action Space}\label{subsec:action-space}
    The action space is the vector of the weights of the $N$ risky assets and one weight is for cash. The weights are bounded between 0 and 1 when the weights (action) are provided by an agent, then for their normalization is used softmax function. The softmax function is defined as:
    \begin{equation}
        \sigma(\bm{w}(t))_i=\frac{e^{\bm{w}(t)_i}}{\sum_{j=1}^{N}e^{\bm{w}(t)_j}}
    \end{equation}
    which is a generalization of the logistic function to multiple dimensions. The softmax function converts a vector of arbitrary real values to a vector of real values in the range (0, 1) that add up to 1. This is useful for representing a probability distribution over $N$ different possible outcomes.

    \subsection{Reward Function}\label{subsec:reward-function}
    The reward function is defined as the rate of portfolio return $\rho(t)$:
    \begin{equation}
        \begin{split}
            \rho(t)&=\frac{v(t)}{v(t-1)}-1\\
            &=\mathbf{w}(t)^\top\mathbf{y}(t)-1\\
        \end{split}
    \end{equation}
    where $\mathbf{w}(t)_{1:N-1}^\top$ represents the new portfolio weights excluding the weight for cash, and $\mathbf{y}(t)$ denotes the estimated price relative vector at time slot $t$. The weight for cash is not considered in the reward function as it has no impact on the portfolio return.

\end{document}
