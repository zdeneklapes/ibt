\documentclass[../xlapes02]{subfiles}



\begin{document}
    \chapter{Experiments and Results}\label{sec:experiments-and-results}
    In this section we present the results of all the experiments performed. We will also discuss any problems that arose during the conduct of these experiments and provide further information on how we selected certain methods and hyperparameters for more extensive testing.

    By performing a series of experiments, we show how our agent performs in a given environment. We compare the best and worst performing models and discuss the results of our experiments with the baseline models and benchmarks in~\cref{sec:experiments}. We also discuss and compare the hyperparameter settings and the advantages and disadvantages of all the datasets used in~\cref{subsec:hyperparameters}. We also provide some information on how we approached performance testing of the proposed models and what technologies were used for the MLOps concepts, such as tracking training and test runs~\cref{sec:wandb}. Finally, we discuss the results of our experiments and provide a summary of our findings~\cref{sec:summary}.
    \\
    \\
    The computer used for experiments has the following specifications:
    \begin{itemize}
        \item Operating System: Ubuntu 20.04.6 LTS (GNU/Linux 5.4.0-146-generic x86\_64)
        \item CPU: 2 x Intel Xeon CPU E5-2620 v3 @ 2.40GHz, each with six cores, for a total of 12 cores.
        \item RAM: 2 x 32 GB RAM running at 2133MHz, using quad-channel architecture for faster memory access
        \item GPU: 4 x NVIDIA GTX 1080 (Pascal) with 8GB RAM each, providing a total of 32GB GPU memory.
    \end{itemize}
    This configuration is well-suited for AI workloads that can benefit from both CPU and GPU parallelism, such as deep learning. The combination of two Intel Xeon CPUs provides a total of 12 cores, which can be utilized for parallel CPU processing. The 64GB of RAM, along with quad-channel architecture, provides fast access to memory, which is essential for large-scale machine learning tasks. The 4 NVIDIA GTX 1080 GPUs provide additional processing power and GPU memory, which can be leveraged by deep learning frameworks such as TensorFlow or PyTorch to accelerate model training and inference.


    \section{Weights \& Biases}\label{sec:wandb}
    Weights \& Biases (W\&B) is a powerful machine learning experiment tracking and visualization tool that helps data scientists and machine learning practitioners manage their experiments. With W\&B, users can log experiment metrics in real-time, track hyperparameters, and compare and reproduce experiments easily. W\&B offers various visualization tools like interactive plots, histograms, and confusion matrices, which help users analyze and understand experiment results.

    \begin{itemize}
        \item Experiment tracking: W\&B allows users to log experiment metrics such as loss, accuracy, and other custom metrics in real-time during training. These metrics are logged to a central dashboard, making it easy to monitor and compare multiple experiments.
        \item Hyperparameter tuning: W\&B supports hyperparameter sweeps, allowing users to explore different hyperparameter configurations in parallel and find optimal hyperparameter settings for their models.
        \item Visualization: W\&B provides a variety of visualization tools, including interactive graphs, histograms, confusion matrices, and more, to help users analyze and understand experimental results. We use some of the W\&B graphs to compare hyperparameters and their values to understand their effect on the overall range of rewards~\crefrange{fig:wb-chart1}{fig:wb-chart3}.
        \item Artifact management: W\&B allows users to log and version datasets, models, and other artifacts, making it easy to track and reproduce experiments with specific data and model versions.
        \item Collaboration: W\&B enables team collaboration by allowing users to share experiment results, visualizations, and artifacts with team members, facilitating communication and collaboration among team members.
    \end{itemize}

    In this paper, for example, training and test runs are publicly recorded on the W\&B website. The results of each run, including the stored models and datasets, can be found at the following URL: \url{https://wandb.ai/investai/portfolio-allocation}. W\&B provides a rich API that offers many additional features not mentioned in this description. For more information, please refer to the W\&B documentation at \url{https://docs.wandb.ai/} for a detailed description of all features and APIs.


    \section{Experiments}\label{sec:experiments}
    All experiments focused on several metrics. The first is whether the model can outperform standard indices such as the S\&P 500, DJIA and others. Second, what dataset is more appropriate for training the model. The third metric is finding the optimal hyperparameters for the model. And the last metric is the time to train and test the model. All experiments will be described in accordance with our testing period, which is from 2017-01-25 to 2022-12-16.

    \subsection{Hyperparameters}\label{subsec:hyperparameters}
    The hyperparameters include various parameters related to the learning algorithm and the environment, such as learning rate, batch size, number of steps, and the coefficients for the value function and entropy.

    \subsubsection{Hyperparameter tuning}
    Hyperparameter tuning is the process of finding the best set of hyperparameters that maximize the performance of a machine learning model. In this case, the hyperparameter sweep was performed on 27 hyperparameters using the wandb (Weights and Biases) sweep, which helps to track and visualize the training process. The hyperparameters were chosen based on their potential impact on the performance of the model and to minimize \emph{train/loss}, and the range of values for each hyperparameter was determined based on prior knowledge and experimentation.

    The sweep was performed with 5 different reinforcement learning algorithms, including A2C, PPO, SAC, DDPG, and TD3, and each algorithm was trained on 3 different datasets. The hyperparameters for each algorithm and dataset were chosen randomly from the given range or values:
    \begin{center}
        \begin{tabular}{|l|c|}
            \hline
            \textbf{Parameter}        & \textbf{Values/range}                           \\ \hline
            learning\_rate            & <0.0001, 0.01>                                  \\ \hline
            n\_steps                  & [32, 64, 128, 256, 512, 1024, 2048]             \\ \hline
            gamma                     & <0.9, 0.999>                                    \\ \hline
            gae\_lambda               & <0.8, 0.999>                                    \\ \hline
            ent\_coef                 & <0.0001, 0.01>                                  \\ \hline
            vf\_coef                  & <0.0001, 0.01>                                  \\ \hline
            max\_grad\_norm           & <0.5, 0.99>                                     \\ \hline
            rms\_prop\_eps            & <0.0001, 0.01>                                  \\ \hline
            sde\_sample\_freq         & <4, 32>                                         \\ \hline
            batch\_size               & [32, 64, 128, 256, 512, 1024, 2048, 4096, 8192] \\ \hline
            n\_epochs                 & <1, 10>                                         \\ \hline
            clip\_range               & <0.1, 0.3>                                      \\ \hline
            clip\_range\_vf           & [None, 0.05, 0.1, 0.15, 0.2]                    \\ \hline
            target\_kl                & <0.01, 0.05>                                    \\ \hline
            buffer\_size              & [1000, 2000, 3000, 4000, 5000]                  \\ \hline
            learning\_starts          & <100, 1000>                                     \\ \hline
            tau                       & <0.001, 0.01>                                   \\ \hline
            train\_freq               & <1, 4>                                          \\ \hline
            gradient\_steps           & <1, 4>                                          \\ \hline
            target\_update\_interval  & <1, 4>                                          \\ \hline
            target\_entropy           & <0.1, 0.2>                                      \\ \hline
            policy\_delay             & <1, 4>                                          \\ \hline
            target\_policy\_noise     & <0.1, 0.2>                                      \\ \hline
            target\_noise\_clip       & <0.1, 0.2>                                      \\ \hline
            exploration\_fraction     & <0.1, 0.2>                                      \\ \hline
            exploration\_initial\_eps & <0.1, 0.2>                                      \\ \hline
            exploration\_final\_eps   & <0.1, 0.2>                                      \\ \hline
        \end{tabular}
    \end{center}

    The training was repeated 10 times for each combination of 5 algorithms, 3 datasets and 10 repetitions (of random hyperparameters) for each combination. The average training and testing time was 1 hour per run, the total training time was 150 hours. The results of the hyperparameter testing are shown in the graphs where we choose several the worst models and several the best models to compare and analyze the performance of the agent, given the hyperparameters~\crefrange{fig:wb-chart1}{fig:wb-chart3}.

    \subsubsection{Best and worst hyperparameters}\label{subsubsec:best-and-worst-hyperparameters}
    The performance is being evaluated based on the \emph{test/total\_reward} attribute. The value of this attribute is being used to distinguish individual training runs, and the color of the curve represents the value of this attribute. The higher the value of \emph{test/total\_reward}, the curve is colored in \textcolor[RGB]{255,128,0}{orange} indicating better performance. On the other hand, the lower the value of \emph{test/total\_reward}, the curve is colored in \textcolor[RGB]{100,0,200}{purple} for poor performance. By examining the graphs presented in~\crefrange{fig:wb-chart1}{fig:wb-chart3}, we can identify the optimal parameters for the agent to achieve peak performance.

    \paragraph{Performance Metric}
    The initial value of \emph{test/total\_reward} is set to $1$, and its value is calculated as the cumulative product of the rewards obtained by the agent during the testing phase, as per equation~\ref{eq:portfolio-value}.

    For instance, if the agent receives rewards of $0.9$ and $1$ in the first step, then the value of \emph{test/total\_reward} would be $0.9 \times 1 = 0.9$ in the second step. If in the second step the agent receives rewards of $1.1$, then the value of \emph{test/total\_reward} would be $0.9 \times 1.1 = 0.99$ in the third step, and so on. Thus, if the final value of \emph{test/total\_reward} is $1.9$, it indicates that the agent received a total reward of $0.9$ over all the steps, which is equivalent to a portfolio return of $90\%$ over the testing period.

    \newpage
    \begin{figure}[h!]
        \centering
        \includegraphics[width=\linewidth, height=0.2\paperheight]{image/wandb/wb1}
        \caption{W\&B Chart of parameters and their performance}
        \label{fig:wb-chart1}
    \end{figure}
    \begin{figure}[h!]
        \centering
        \includegraphics[width=\linewidth, height=0.2\paperheight]{image/wandb/wb1}
        \caption{W\&B Chart of parameters and their performance}
        \label{fig:wb-chart2}
    \end{figure}
    \begin{figure}[h!]
        \centering
        \includegraphics[width=\linewidth, height=0.2\paperheight]{image/wandb/wb1}
        \caption{W\&B Chart of parameters and their performance}
        \label{fig:wb-chart3}
    \end{figure}

    The following table shows the hyperparameters of the 2 best and 2 worst trained models~\cref{tab:best-worst-hyperparameters}:
    \begin{table}[!ht]
        \centering
        \label{tab:best-worst-hyperparameters}
        \begin{tabular}{|l||l|l||l|l|}
            \hline
            \textbf{Run ID}                    & p3irnh80                                     & 8tml2ozg                                     & zfjr0ks0                                     & pky1wslb                                     \\ \hline
            \textbf{algo}                      & DDPG                                         & A2C                                          & A2C                                          & PPO                                          \\ \hline
            \textbf{learning\_rate}            & 0.00685                                      & 0.00671                                      & 0.00698                                      & 0.00471                                      \\ \hline
            \textbf{n\_steps}                  & 256                                          & 128                                          & 128                                          & 32                                           \\ \hline
            \textbf{gamma}                     & 0.9294                                       & 0.94108                                      & 0.93858                                      & 0.9289                                       \\ \hline
            \textbf{gae\_lambda}               & 0.99624                                      & 0.90203                                      & 0.9383                                       & 0.85115                                      \\ \hline
            \textbf{ent\_coef}                 & 0.00203                                      & 0.0013                                       & 0.00525                                      & 0.00173                                      \\ \hline
            \textbf{vf\_coef}                  & 0.0082                                       & 0.00683                                      & 0.00125                                      & 0.00527                                      \\ \hline
            \textbf{max\_grad\_norm}           & 0.71475                                      & 0.56558                                      & 0.94281                                      & 0.77201                                      \\ \hline
            \textbf{rms\_prop\_eps}            & 0.00024                                      & 0.00435                                      & 0.0075                                       & 0.0067                                       \\ \hline
            \textbf{sde\_sample\_freq}         & 30                                           & 26                                           & 23                                           & 15                                           \\ \hline
            \textbf{batch\_size}               & 128                                          & 64                                           & 8192                                         & 64                                           \\ \hline
            \textbf{n\_epochs}                 & 5                                            & 8                                            & 7                                            & 8                                            \\ \hline
            \textbf{clip\_range}               & 0.18938                                      & 0.19559                                      & 0.2274                                       & 0.22578                                      \\ \hline
            \textbf{clip\_range\_vf}           & 0.05                                         & ~                                            & 0.15                                         & 0.05                                         \\ \hline
            \textbf{target\_kl}                & 0.02394                                      & 0.04957                                      & 0.03176                                      & 0.03337                                      \\ \hline
            \textbf{buffer\_size}              & 2000                                         & 5000                                         & 2000                                         & 2000                                         \\ \hline
            \textbf{learning\_starts}          & 387                                          & 163                                          & 569                                          & 165                                          \\ \hline
            \textbf{tau}                       & 0.00815                                      & 0.00461                                      & 0.00269                                      & 0.00469                                      \\ \hline
            \textbf{train\_freq}               & 2                                            & 2                                            & 3                                            & 2                                            \\ \hline
            \textbf{gradient\_steps}           & 1                                            & 3                                            & 1                                            & 3                                            \\ \hline
            \textbf{target\_update\_interval}  & 3                                            & 2                                            & 4                                            & 2                                            \\ \hline
            \textbf{target\_entropy}           & 0.19013                                      & 0.15826                                      & 0.18233                                      & 0.13979                                      \\ \hline
            \textbf{policy\_delay}             & 2                                            & 4                                            & 2                                            & 3                                            \\ \hline
            \textbf{target\_policy\_noise}     & 0.11459                                      & 0.15052                                      & 0.10674                                      & 0.12649                                      \\ \hline
            \textbf{target\_noise\_clip}       & 0.18538                                      & 0.10527                                      & 0.18672                                      & 0.14197                                      \\ \hline
            \textbf{exploration\_fraction}     & 0.12082                                      & 0.10871                                      & 0.16125                                      & 0.16269                                      \\ \hline
            \textbf{exploration\_final\_eps}   & 0.18574                                      & 0.10366                                      & 0.10618                                      & 0.19131                                      \\ \hline
            \textbf{exploration\_initial\_eps} & 0.16609                                      & 0.10003                                      & 0.16434                                      & 0.1039                                       \\ \hline
            \textbf{test/total\_reward}        & \textcolor[RGB]{50,150,50}{\textbf{1.97214}} & \textcolor[RGB]{50,150,50}{\textbf{1.97061}} & \textcolor[RGB]{150,50,50}{\textbf{1.62093}} & \textcolor[RGB]{150,50,50}{\textbf{1.63348}} \\ \hline
        \end{tabular}
        \caption{Best and worst model and their training hyperparameters. All hyperparameters are also available online URL: \url{https://wandb.ai/investai/portfolio-allocation} with much more details and experiments.}
    \end{table}

    \subsection{Datasets}\label{subsec:datasets}
    In the previous chapter we compare models based on the \emph{test/total\_reward} of the agent, comparing training hyperparameters Now we look and compare model performance based on the dataset used. We compare the same models (2 best and 2 worst models) used in the previous section in~\cref{tab:best-worst-datasets}, and we also include the best and worst model from the \emph{Fundamental Analysis} dataset and \emph{Technical Analysis} in~\cref{TODO}. The results are shown in Table~\ref{tab:best-worst-datasets}.
    \begin{table}[!ht]
        \centering
        \label{tab:best-worst-datasets}
        \begin{tabular}{|l|l|l|}
            \hline
            \textbf{Run ID} & \textbf{Dataset Name} & \textbf{test/total\_reward}                \\ \hline
            p3irnh80        & Combined              & \textcolor[RGB]{50,150,50}{\textbf{1.972}} \\ \hline
            8tml2ozg        & Combined              & \textcolor[RGB]{50,150,50}{\textbf{1.970}} \\ \hline
            zfjr0ks0        & Fundamental Analysis  & \textcolor[RGB]{150,50,50}{\textbf{1.620}} \\ \hline
            pky1wslb        & Combined              & \textcolor[RGB]{150,50,50}{\textbf{1.633}} \\ \hline
        \end{tabular}
        \caption{Permormance of the datasets used for training the two best and two worst models used in the previous section for comparing hyperparameters as well.}
    \end{table}

    \begin{table}[!ht]
        \centering
        \label{tab:worst-datasets}
        \begin{tabular}{|l|l|l|}
            \hline
            \textbf{Run ID} & \textbf{Dataset Name} & \textbf{test/total\_reward}                \\ \hline
            zfjr0ks0        & Fundamental Analysis  & \textcolor[RGB]{150,50,50}{\textbf{1.620}} \\ \hline
            pky1wslb        & Combined              & \textcolor[RGB]{150,50,50}{\textbf{1.633}} \\ \hline
            2161deh4        & Technical Analysis    & \textcolor[RGB]{150,50,50}{\textbf{1.642}} \\ \hline
            ipl1v8io        & Technical Analysis    & \textcolor[RGB]{150,50,50}{\textbf{1.643}} \\ \hline
            2161deh4        & Technical Analysis    & \textcolor[RGB]{150,50,50}{\textbf{1.642}} \\ \hline
            2161deh4        & Technical Analysis    & \textcolor[RGB]{150,50,50}{\textbf{1.642}} \\ \hline
        \end{tabular}
        \caption{The datasets with the poorest performing models.}
    \end{table}

    \begin{table}[!ht]
        \centering
        \label{tab:best-datasets}
        \begin{tabular}{|l|l|l|}
            \hline
            \textbf{Run ID} & \textbf{Dataset Name} & \textbf{test/total\_reward}                \\ \hline
            p3irnh80        & Combined              & \textcolor[RGB]{50,150,50}{\textbf{1.972}} \\ \hline
            8tml2ozg        & Combined              & \textcolor[RGB]{50,150,50}{\textbf{1.970}} \\ \hline
            geaioz9h        & Fundamental Analysis  & \textcolor[RGB]{50,150,50}{\textbf{1.965}} \\ \hline
            8iq9e37s        & Combined              & \textcolor[RGB]{50,150,50}{\textbf{1.956}} \\ \hline
            y3zz2sv3        & Combined              & \textcolor[RGB]{50,150,50}{\textbf{1.955}} \\ \hline
            4qr3nk43        & Fundamental Analysis  & \textcolor[RGB]{50,150,50}{\textbf{1.945}} \\ \hline
        \end{tabular}
        \caption{The datasets with the best performing models.}
    \end{table}

    Based on the~\crefrange{tab:best-worst-datasets}{tab:best-datasets} it appears that the performance of the models is highly dependent on the choice of hyperparameters, making it difficult to determine the worst dataset. However, it seems clear that when the hyperparameters are well-tuned, the \emph{Combined Dataset} performs the best, while the \emph{Fundamental Dataset} also performs well.

    All the information, including hyperparameters, datasets, and other details, are available online at the following URL: \url{https://wandb.ai/investai/portfolio-allocation}. The website contains extensive information about the experiments and includes many more details.

    \subsection{Baselines and Backtesting}\label{subsec:baselines-and-backtesting}
    In this subsubsection, we compare the performance of the agent to several baselines, including well-known market indexes and investment strategies. The following strategies and indexes were used as baselines:

    \begin{itemize}
        \item \textbf{The S\&P 500 Index (GSPC)} is a capitalization-weighted index that tracks the performance of 500 large publicly traded companies in the US. The index is maintained by S\&P Dow Jones Indices and is one of the most widely followed equity indices.
        \item \textbf{The Dow Jones Industrial Average (DJIA)} is a price-weighted index that tracks the stock performance of 30 large companies listed on US stock exchanges. It is also one of the most commonly followed equity indices.
        \item \textbf{The Russell 2000 (RUT)} is a stock market index that tracks the performance of 2,000 small-cap companies in the US. It is a subset of the Russell 3000 index, which represents about 98\% of the total market capitalization of the US equity market.
        \item \textbf{The NASDAQ Composite (IXIC)} is a stock market index that tracks the performance of all the companies listed on the NASDAQ stock exchange, which is primarily composed of technology and growth-oriented companies. It is one of the most widely-followed stock market indices in the world.
        \item \textbf{Minimum Variance} is a portfolio allocation strategy that aims to minimize the variance of the portfolio's returns. The strategy assumes that the variance of the portfolio's returns is a measure of risk~\cite{investopedia-portfolio-variance}.
        \item \textbf{Maximum Sharpe Ratio} is a portfolio allocation strategy that aims to maximize the Sharpe ratio of the portfolio. The strategy assumes that the Sharpe ratio is a measure of risk-adjusted return~\cite{investopedia-sharpe-ratio}.
    \end{itemize}

    \subsubsection{Cumulative Returns}
    The figure shown in~\cref{fig:cumulative_return} demonstrates that appropriate hyperparameter tuning can lead to successful results from a model, whereas unsuitable hyperparameters can result in unsatisfactory outcomes. The model that performs the best is trained with optimal hyperparameters and surpasses standard indexes such as DJI, RUT, IXIC, and S\&P500, albeit not consistently throughout the testing period. Towards the end of the testing period, the model outperforms S\&P500 for a brief period of time.
    \begin{figure}[h!]
        \centering
        \begin{subfigure}[t]{\experimentimgwidth\textwidth}
            \centering
            \includegraphics[width=\linewidth]{image/figure/returns_max}
            \caption{The model with the highest \emph{text/total\_reward}}
            \label{fig:returns_max}
        \end{subfigure}
        \hfill
        \begin{subfigure}[t]{\experimentimgwidth\textwidth}
            \centering
            \includegraphics[width=\linewidth]{image/figure/returns_min}
            \caption{The model with the smallest \emph{text/total\_reward}}
            \label{fig:returns_min}
        \end{subfigure}
        \caption{Cumulative return.}
        \label{fig:cumulative_return}
    \end{figure}

    \subsubsection{Drawdowns}
    By examining the draw-down graphs in \cref{fig:drawdown}, it is evident that our tuned model is better than the DJI index for larger drawdowns (around $6\%$), although both of our models perform well in comparison to the DJI index.
    \begin{figure}[h!]
        \begin{subfigure}[t]{\experimentimgwidth\textwidth}
            \centering
            \includegraphics[width=\linewidth]{image/figure/drawdown_underwater_max}
            \caption{The model with the highest \emph{text/total\_reward}}
            \label{fig:drawdown_underwater_max}
        \end{subfigure}
        \hfill
        \begin{subfigure}[t]{\experimentimgwidth\textwidth}
            \centering
            \includegraphics[width=\linewidth]{image/figure/drawdown_underwater_dji}
            \caption{The DJI index}
            \label{fig:drawdown_underwater_dji}
        \end{subfigure}
        \caption{Drawdown.}
        \label{fig:drawdown}
    \end{figure}

    \subsubsection{Monthly and Annual Returns}
    In figure \cref{fig:month_annual_returns}, both the monthly and annual returns of the models are presented, allowing for easy comparison of their performance. The results are similar to the cumulative returns described in the previous subsubsection, but the monthly returns provide a clearer understanding of when the models performed well or poorly. The model with the highest \emph{text/total\_reward} consistently outperforms the DJI index in most months, while even the model with the lowest \emph{text/total\_reward} performs better than the DJI index in the majority of months.

    Notably, the models achieved significant outperformance during the big drawdown period, as shown in the heatmaps in subfigures \cref{subfig:montly_returns_heatmap_max} and \cref{subfig:montly_returns_heatmap_dji}. Specifically, the model with the highest \emph{text/total\_reward} performed better than the DJI index during the COVID-19 pandemic months of February, March and April 2020, due to the differing company weightings in the DJI index. This suggests that our model was able to learn which companies with specific features were better to hold during this period.

    While there were some months where the DJI index performed better than our models, such as in April and October 2022, the performance difference was not significant. Therefore, we can conclude that our models performed well in the majority of months, as demonstrated by the annual graphs in subfigures \cref{subfig:annual_returns_max} and \cref{subfig:annual_returns_dji}.

    \begin{figure}[h!]
        \centering
        %
        \begin{subfigure}[t]{\experimentimgwidth\textwidth}
            \centering
            \label{subfig:montly_returns_heatmap_max}
            \includegraphics[width=\linewidth]{image/figure/monthly_returns_heatmap_max}
            \caption{The model with the highest \emph{text/total\_reward}}
        \end{subfigure}
        \hfill
        \begin{subfigure}[t]{\experimentimgwidth\textwidth}
            \centering
            \label{subfig:annual_returns_max}
            \includegraphics[width=\linewidth]{image/figure/annual_returns_max}
            \caption{The model with the highest \emph{text/total\_reward}}
        \end{subfigure}

        %
        \begin{subfigure}[t]{\experimentimgwidth\textwidth}
            \centering
            \includegraphics[width=\linewidth]{image/figure/monthly_returns_heatmap_min}
            \caption{The model with the smallest \emph{text/total\_reward}}
        \end{subfigure}
        \hfill
        \begin{subfigure}[t]{\experimentimgwidth\textwidth}
            \centering
            \includegraphics[width=\linewidth]{image/figure/annual_returns_min}
            \caption{The model with the smallest \emph{text/total\_reward}}
        \end{subfigure}

        %
        \begin{subfigure}[t]{\experimentimgwidth\textwidth}
            \centering
            \label{subfig:montly_returns_heatmap_dji}
            \includegraphics[width=\linewidth]{image/figure/monthly_returns_heatmap_dji}
            \caption{The DJI index}
        \end{subfigure}
        \hfill
        \begin{subfigure}[t]{\experimentimgwidth\textwidth}
            \centering
            \label{subfig:annual_returns_dji}
            \includegraphics[width=\linewidth]{image/figure/annual_returns_dji}
            \caption{The DJI index}
        \end{subfigure}

        %
        \caption{The monthly returns.}
        \label{fig:month_annual_returns}
    \end{figure}

    \subsubsection{Overall Performance and Statistics}
%    TODO: Table performance with stats from pyfolio
    \begin{table}[!ht]
        \centering
        \begin{tabular}{|l|l|l|l|l|}
            \hline
            \textbf{Metric}              & \textbf{Min Total Reward} & \textbf{Max Total Reward} & \textbf{DJI} & \textbf{AI4Finance} \\ \hline
            \textbf{Annual return}       & 0.085                     & 0.122                     & 0.106        & 0.09                \\ \hline
            \textbf{Cumulative returns}  & 0.621                     & 0.972                     & 0.81         & 0.28                \\ \hline
            \textbf{Annual volatility}   & 0.181                     & 0.188                     & 0.193        & 0.232               \\ \hline
            \textbf{Sharpe ratio}        & 0.543                     & 0.706                     & 0.619        & 0.49                \\ \hline
            \textbf{Calmar ratio}        & 0.288                     & 0.391                     & 0.336        & 0.24                \\ \hline
            \textbf{Stability}           & 0.855                     & 0.913                     & 0.863        & 0.04                \\ \hline
            \textbf{Max drawdown}        & -0.297                    & -0.312                    & -0.316       & -0.375              \\ \hline
            \textbf{Omega ratio}         & 1.118                     & 1.158                     & 1.134        & 1.14                \\ \hline
            \textbf{Sortino ratio}       & 0.763                     & 0.994                     & 0.868        & 0.67                \\ \hline
            \textbf{Skew}                & -0.249                    & -0.312                    & -0.357       & ~                   \\ \hline
            \textbf{Kurtosis}            & 16.819                    & 19.569                    & 17.3         & ~                   \\ \hline
            \textbf{Tail ratio}          & 0.883                     & 0.883                     & 0.93         & 1.03                \\ \hline
            \textbf{Daily value at risk} & -0.022                    & -0.023                    & -0.024       & -0.028              \\ \hline
            \textbf{Beta}                & 0.879                     & 0.921                     & 0.943        & 0.68                \\ \hline
            \textbf{Alpha}               & 0.02                      & 0.02                      & 0.02         & 0.03                \\ \hline
        \end{tabular}
        \caption{Performance statistics. AI4Finance Agent statistics were taken from the github repository~\cite{FinRL-Tutorials}.}
    \end{table}


    \section{Summary}\label{sec:summary}
    TODO


\end{document}
