\documentclass[../xlapes02]{subfiles}
\begin{document}
    \chapter{Conclusions}\label{ch:conclusions}
    The aim of this thesis was to develop a profitable portfolio allocation strategy using deep reinforcement learning, and we successfully achieved this goal as outlined in Chapter 1.

    One of the main contributions of this work is the creation of three datasets that describe the environment for the agent, as well as the implementation of a training and testing pipeline that is connected to the Weights and Biases platform.\ We were able to identify the most appropriate dataset for training the agent, which consists of a combination of fundamental and technical analysis features.\ We found that technical analysis alone is not sufficient to predict future stock prices, but it can serve as a useful indicator for the agent to learn the best strategy.\ In contrast, fundamental analysis provides a good enough description of the agent's world to predict stock prices without any additional factors.

    Furthermore, we independently configured hyperparameters for each common RL algorithm and dataset and made them publicly available through Weights and Biases.\ This transparency and reproducibility of our work make it easier for other researchers to replicate and improve upon our results.

    Although our agent's performance could not match the annual returns of investors like Warren Buffett or Peter Lynch, who consistently achieve over 20\%, we were able to generate an annual return of over 12\%.\ Based on our model's returns, there is considerable room for improvement, which could enable us to surpass even these legendary investors.

    We found that our agent could learn the best strategy even in the case of the COVID-19 pandemic and other major drops in the stock market.\ Specifically, the agent with the lowest cumulative return was more efficient during such events, while the agent with the highest cumulative return was more effective during draw-downs and skyrocketing markets.

    Overall, our RL-based approach for portfolio allocation holds great potential, and we suggest future work to further enhance our results by incorporating additional factors such as sentiment analysis, macroeconomic factors, and industry-specific factors.\ These enhancements could enable us to train more accurate RL agents that surpass even the top investors in the field.\ It will be intriguing to observe how these factors can be implemented in real-time market scenarios to predict market trends and facilitate investment decisions.

    Finally, we believe that our work can inspire further research in this area.\ In conclusion, our findings, along with the availability of data and computational resources, suggest that RL-based portfolio allocation is a promising direction for future investment research.
\end{document}
