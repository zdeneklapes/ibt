% Informace o práci/projektu / Information about the thesis
%---------------------------------------------------------------------------
\projectinfo{
%
%Prace / Thesis
    project={BP},            %typ práce BP/SP/DP/DR  / thesis type (SP = term project)
    year={2023},             % rok odevzdání / year of submission
    date=\today,             % datum odevzdání / submission date
%
%Nazev prace / thesis title
    title.cs={Využití zpětnovazebného učení pro automatickou alokaci akciového portfolia},  %  thesis title in czech language (according to assignment)
    title.en={Reinforcement Learning for Automated Stock Portfolio Allocation}, %  thesis title in english
    title.length={14.5cm}, %  setting the length of a block with a thesis title for adjusting a line break (can be defined here or below)
    sectitle.length={14.5cm}, %  setting the length of a block with a second thesis title for adjusting a line break (can be defined here or below)
    dectitle.length={14.5cm}, %  setting the length of a block with a thesis title above declaration for adjusting a line break (can be defined here or below)
%
%Autor / Author
    author.name={Zdeněk},   % jméno autora / author name
    author.surname={Lapeš},   % příjmení autora / author surname
%
%Ustav / Department
    department={UITS}, % UIFS/UITS/UPGM / fill in appropriate abbreviation of the department according to assignment: UPSY/UIFS/UITS/UPGM
%
% Školitel / supervisor
    supervisor.name={Milan},   % jméno školitele / supervisor name
    supervisor.surname={Češka},   % příjmení školitele / supervisor surname
    supervisor.title.p={doc. RNDr.},   %titul před jménem (nepovinné) / title before the name (optional)
    supervisor.title.a={Ph.D.},    %titul za jménem (nepovinné) / title after the name (optional)
%
% Klíčová slova / keywords
    keywords.cs={
        umělá inteligence, AI, posilované učení, alokace akciového portfolia,
        moderni teorie portfolia, Q-learning, neuronové sítě, akciový trh
    },
    keywords.en={
        artificial intelligence, AI, reinforcement learning, stock portfolio allocation,
        modern portfolio theory, Q-learning, neural networks, stock market
    },
%
% Abstrakt / Abstract
    abstract.cs={
        Alokace portfolia se zaměřuje na výběr souboru aktiv, do kterých investujete.
        Cílem je maximalizovat očekávaný výnos pro danou míru rizika
        pro námi zvolený investiční horizont.
        V dřívějších dobách tohle proces je obvykle prováděn ručně zkušeným investorem.
        V dnešní době existuje mnoho metod alokace portfolia, které nejsou
        úspěšné v reálném světě nebo je lze zlepšit a potenciál současných
        technologií není na finančním trhu plně prozkoumán.
        Pro řešení problému alokace portfolia navrhuji metody
        založené na posilovacím učení.
        Agent se naučí přesné strategie výběru aktiv a jejich váhu pro portfolio,
        na základě kterého by je lidský expert vybíral,
        jako je fundamentální analýza a zdraví společnosti.
        Diplomová práce se zabývá problémem alokace portfolia
        pomocí posilovacího učení, které pomáhá při výběru nejlepších aktiv
        a jejich důležitosti pro portfolio.
    }, % abstrakt v českém či slovenském jazyce / abstract in czech or slovak language
    abstract.en={
        Portfolio allocation is about selecting a set of assets to invest in.
        The goal is to maximize the expected return for
        a given level of risk for an investing horizon selected by us.
        In former times, this the process is usually done manually
        by an expert investor.
        Nowadays, there are many portfolio allocation methods that are not
        successful in the real world or can be improved and the
        potential of current technologies is not fully explored
        in the financial market.
        To solve the problem of portfolio allocation, I propose a methods
        based on reinforcement learning.
        The agent will learn the exact strategies of
        selecting assets and their weight for the portfolio based on
        which the human expert would select it,
        like fundamental analysis and the health of the company.
        The thesis deals with the problem of portfolio allocation
        using reinforcement learning, which helps in selecting
        the best assets and their importance for the portfolio.
    }, % abstrakt v anglickém jazyce / abstract in english
%  Declaration (for thesis in english should be in english; for project practice can be commented out)
    declaration={
        I hereby declare that this Bachelor's thesis was prepared as an
        original work by the author under the supervision
        of Mr. Milan Češka, Ph.D. % TODO: make this dynamic
        I have listed all the literary sources, publications and other sources,
        which were used during the preparation of this thesis.
    },
%
%  Acknowledgement (optional, ideally in the language of the thesis)
    acknowledgment={
        I would like to thank my supervisor, Mr. Milan Češka, for his guidance
        and support during the preparation of this thesis.
        I would also like to thank my family and friends for their support.
    },
%
%  Extended abstract (approximately 3 standard pages) - can be defined here or below
%    extendedabstract={ % path must be like that because this is inserted directly into xlapes02.tex
%        %%%%%%%%%%%%%%%%%%%%%%%%%%%%%%%%%%%%%%%%%%%%%%%%%%%%%%%%%%%%%%%%%%%%%%%%%%%%%%%
%%% První část – Jaký se řeší problém? Jaké je téma? Jaký je cíl textu?
%%%%%%%%%%%%%%%%%%%%%%%%%%%%%%%%%%%%%%%%%%%%%%%%%%%%%%%%%%%%%%%%%%%%%%%%%%%%%%%

This work is focused on the Reinforcement Learning with Application in Finance
especially in the Stock Portfolio Allocation.
The goal of this project is to implement an agent that with specified
Deep Reinforcement Learning Algorithm and properly implemented
Markov Decision Process will be able to allocate the portfolio in the
most efficient way.

%%%%%%%%%%%%%%%%%%%%%%%%%%%%%%%%%%%%%%%%%%%%%%%%%%%%%%%%%%%%%%%%%%%%%%%%%%%%%%%
%%% Druhá část – Jak je problém vyřešen? Cíl naplněn?
%%%%%%%%%%%%%%%%%%%%%%%%%%%%%%%%%%%%%%%%%%%%%%%%%%%%%%%%%%%%%%%%%%%%%%%%%%%%%%%

As programming language I have chosen Python.
The strategies are implemented in the OpenAI Gym environment
with superstructure focused on Finance Environment
provided by the FinRL-Meta framework.
% TODO

%%%%%%%%%%%%%%%%%%%%%%%%%%%%%%%%%%%%%%%%%%%%%%%%%%%%%%%%%%%%%%%%%%%%%%%%%%%%%%%
%%% Třetí část – Jaké jsou konkrétní výsledky? Jak dobře je problém vyřešen?
%%%%%%%%%%%%%%%%%%%%%%%%%%%%%%%%%%%%%%%%%%%%%%%%%%%%%%%%%%%%%%%%%%%%%%%%%%%%%%%


%%%%%%%%%%%%%%%%%%%%%%%%%%%%%%%%%%%%%%%%%%%%%%%%%%%%%%%%%%%%%%%%%%%%%%%%%%%%%%%
%%% Čtvrtá část – Takže co? Čím je to užitečné vědě a čtenáři?
%%%%%%%%%%%%%%%%%%%%%%%%%%%%%%%%%%%%%%%%%%%%%%%%%%%%%%%%%%%%%%%%%%%%%%%%%%%%%%%

%    },
%
    faculty={FIT}, % FIT/FEKT/FSI/FA/FCH/FP/FAST/FAVU/USI/DEF
}
