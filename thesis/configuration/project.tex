% Informace o práci/projektu / Information about the thesis
%---------------------------------------------------------------------------
\projectinfo{
%
%Prace / Thesis
    project={BP},            %typ práce BP/SP/DP/DR  / thesis type (SP = term project)
    year={2023},             % rok odevzdání / year of submission
    date=\today,             % datum odevzdání / submission date
%
%Nazev prace / thesis title
    title.cs={Využití zpětnovazebného učení pro automatickou alokaci akciového portfolia},  %  thesis title in czech language (according to assignment)
    title.en={Reinforcement Learning for Automated Stock Portfolio Allocation}, %  thesis title in english
    title.length={14.5cm}, %  setting the length of a block with a thesis title for adjusting a line break (can be defined here or below)
    sectitle.length={14.5cm}, %  setting the length of a block with a second thesis title for adjusting a line break (can be defined here or below)
    dectitle.length={14.5cm}, %  setting the length of a block with a thesis title above declaration for adjusting a line break (can be defined here or below)
%
%Autor / Author
    author.name={Zdeněk},   % jméno autora / author name
    author.surname={Lapeš},   % příjmení autora / author surname
%
%Ustav / Department
    department={UITS}, % UIFS/UITS/UPGM / fill in appropriate abbreviation of the department according to assignment: UPSY/UIFS/UITS/UPGM
%
% Školitel / supervisor
    supervisor.name={Milan},   % jméno školitele / supervisor name
    supervisor.surname={Češka},   % příjmení školitele / supervisor surname
    supervisor.title.p={doc. RNDr.},   %titul před jménem (nepovinné) / title before the name (optional)
    supervisor.title.a={Ph.D.},    %titul za jménem (nepovinné) / title after the name (optional)
%
% Klíčová slova / keywords
    keywords.cs={
        umělá inteligence, AI, posilované učení, alokace akciového portfolia, moderni teorie portfolia, Q-learning, neuronové sítě, akciový trh
    },
    keywords.en={
        artificial intelligence, AI, reinforcement learning, stock portfolio allocation, modern portfolio theory, Q-learning, neural networks, stock market
    },
%
% Abstrakt / Abstract
    abstract.cs={
        Tato práce je zaměřena na téma posilovacího učení aplikovaného na úlohu alokace portfolia. K dosažení tohoto cíle práce nejprve uvádí přehled základní teorie, která zahrnuje nejnovější metody založené na hodnotách a politikách. Následně je v práci popsáno prostředí portfolia Stock a nakonec jsou uvedeny podrobnosti o experimentu a implementaci. Podrobně je rozebrána tvorba datových souborů a její zdůvodnění a metodika. RL agent je poté vycvičen a otestován na třech datových sadách a získané výsledky jsou slibné a překonávají běžné benchmarky. Bylo však zjištěno, že roční výnos agenta stále není lepší než výnosy generované nejlepšími světovými investory. Pipeline byla implementována v jazyce Python 3.10 a ke sledování všech datových sad, modelů a hyperparametrů byla použita technologie Weights \& Biases. Závěrem lze říci, že tato práce představuje významný krok vpřed ve vývoji efektivnějších RL agentů pro finanční investice, kteří mají potenciál překonat i výkonnost nejlepších světových investorů.
    }, % abstrakt v českém či slovenském jazyce / abstract in czech or slovak language
    abstract.en={
        This thesis is focused on the topic of reinforcement learning applied to a task of portfolio allocation. To accomplish this objective, the thesis first presents an overview of the fundamental theory, which includes the latest value-based and policy-based methods. Following that, the thesis describes the Stock portfolio environment, and finally, the experimental and implementation details are presented. The creation of datasets is discussed in detail, along with the rationale and methodology behind it. The RL agent is then trained and tested on three datasets, and the results obtained are promising and outperform common benchmarks. However, it was discovered that the annual return of the agent is still not better than the returns generated by the world's top investors. The pipeline was implemented in Python 3.10, and technology from Weights \& Biases was used to monitor all datasets, models, and hyperparameters. In conclusion, this work represents a significant step forward in the development of more effective RL agents for financial investments, with the potential to exceed even the performance of the world's greatest investors.
    }, % abstrakt v anglickém jazyce / abstract in english
%  Declaration (for thesis in english should be in english; for project practice can be commented out)
    declaration={
        I hereby declare that this Bachelor's thesis was prepared as an original work by the author under the supervision of Mr. Milan Češka, Ph.D.. I have listed all the literary sources, publications and other sources, which were used during the preparation of this thesis.
    },
%
%  Acknowledgement (optional, ideally in the language of the thesis)
    acknowledgment={
        I would like to thank my supervisor, Mr. Milan Češka, for his guidance and support during the preparation of this thesis. I would also like to thank my family and friends for their support.
    },
%
%  Extended abstract (approximately 3 standard pages) - can be defined here or below
%    extendedabstract={ % path must be like that because this is inserted directly into xlapes02.tex
%        %%%%%%%%%%%%%%%%%%%%%%%%%%%%%%%%%%%%%%%%%%%%%%%%%%%%%%%%%%%%%%%%%%%%%%%%%%%%%%%
%%% První část – Jaký se řeší problém? Jaké je téma? Jaký je cíl textu?
%%%%%%%%%%%%%%%%%%%%%%%%%%%%%%%%%%%%%%%%%%%%%%%%%%%%%%%%%%%%%%%%%%%%%%%%%%%%%%%

This work is focused on the Reinforcement Learning with Application in Finance
especially in the Stock Portfolio Allocation.
The goal of this project is to implement an agent that with specified
Deep Reinforcement Learning Algorithm and properly implemented
Markov Decision Process will be able to allocate the portfolio in the
most efficient way.

%%%%%%%%%%%%%%%%%%%%%%%%%%%%%%%%%%%%%%%%%%%%%%%%%%%%%%%%%%%%%%%%%%%%%%%%%%%%%%%
%%% Druhá část – Jak je problém vyřešen? Cíl naplněn?
%%%%%%%%%%%%%%%%%%%%%%%%%%%%%%%%%%%%%%%%%%%%%%%%%%%%%%%%%%%%%%%%%%%%%%%%%%%%%%%

As programming language I have chosen Python.
The strategies are implemented in the OpenAI Gym environment
with superstructure focused on Finance Environment
provided by the FinRL-Meta framework.
% TODO

%%%%%%%%%%%%%%%%%%%%%%%%%%%%%%%%%%%%%%%%%%%%%%%%%%%%%%%%%%%%%%%%%%%%%%%%%%%%%%%
%%% Třetí část – Jaké jsou konkrétní výsledky? Jak dobře je problém vyřešen?
%%%%%%%%%%%%%%%%%%%%%%%%%%%%%%%%%%%%%%%%%%%%%%%%%%%%%%%%%%%%%%%%%%%%%%%%%%%%%%%


%%%%%%%%%%%%%%%%%%%%%%%%%%%%%%%%%%%%%%%%%%%%%%%%%%%%%%%%%%%%%%%%%%%%%%%%%%%%%%%
%%% Čtvrtá část – Takže co? Čím je to užitečné vědě a čtenáři?
%%%%%%%%%%%%%%%%%%%%%%%%%%%%%%%%%%%%%%%%%%%%%%%%%%%%%%%%%%%%%%%%%%%%%%%%%%%%%%%

%    },
%
    faculty={FIT}, % FIT/FEKT/FSI/FA/FCH/FP/FAST/FAVU/USI/DEF
}
